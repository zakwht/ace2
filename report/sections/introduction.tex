\newpage
\section{Introduction}

For over two years, the COVID-19 pandemic has been a hot-topic of research as it inescapably invades our world. COVID-19 is a disease caused by severe acute respiratory syndrome coronavirus 2 (SARS-CoV-2), a virus believed have zoonotic origins \cite{Andersen}.

Recent animal studies have identified a range of mammal species -- including some nonhuman primates, ferrets, hamsters, and bats \cite{Oude} -- that can be infected by and can transmit the virus. Conversely, certain experiments have supported the conclusion that some other species (e.g. the domesticated pig) are explicitly insusceptible to the virus \cite{Oude}\cite{Sreenivasan}. This disjunction promotes the question, \emph{what makes some species susceptible to the virus and others immune?}

The answer might correspond to how the the virus interacts with its target protein in different hosts. The SARS-CoV-2 spike protein's target is the angiotensin-converting enzyme 2 (ACE2). Differences in the ACE2 sequence can affect the spike protein's binding affinity, thus affecting the host's susceptibility to the virus.

The objective of this research is to computationally identify mutations that may inhibit the spike protein's interaction with the ACE2 protein in a host.

\subsection{Related Work}

\textcite{Li} studied the effects that changes in ACE2 protein had on the binding affinity of the SARS-CoV spike protein. By introducing residues of insusceptible ACE2 proteins into susceptible ones, they identified which mutations abolished interaction with the spike protein. Specifically, mutations D31, A41, 82-84NFS, A353, H353, and A357 restricted the interaction.

\textcite{Liu} analyzed the structure of the human ACE2 protein to identify critical residues and binding sites; they concluded the amino acids at positions 31, 35, 38, 82, and 353 were crucial for interaction with the spike protein.